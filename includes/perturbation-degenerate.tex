%%%%%%%%%%%%%%%%%%%%%%%%%%%%%%%%%%%%%%%%%
%
% (c) 2018 by Jennifer Laaser
%
% This work is licensed under the Creative Commons Attribution-NonCommercial-ShareAlike 4.0 International License. To view a copy of this license, visit http://creativecommons.org/licenses/by-nc-sa/4.0/ or send a letter to Creative Commons, PO Box 1866, Mountain View, CA 94042, USA.
%
% The current source for these materials is accessible on Github: https://github.com/jlaaser/quantum-exercises
%
%%%%%%%%%%%%%%%%%%%%%%%%%%%%%%%%%%%%%%%%%

\section*{Degenerate Perturbation Theory\sectionmark{Exercise: Degenerate Perturbation Theory}}

	As discussed in class, the approach described in the previous exercise fails when any of the zeroth-order states are degenerate. In this exercise, we'll consider how to transform our basis set to solve this problem.

	\begin{questions}
	
		\question 
		\begin{parts}
			\part Suppose the states $\{\ket{i^{(0)}}\}$ are degenerate, i.e. $\hat H^{(0)}\ket{i^{(0)}} = E \ket{i^{(0)}}$ for all of the states with different $i$.
			Calculate $H^{(0)}\ket{n^{(0)}}$, where $\ket{n^{(0)}}$ is a linear combination of these states, e.g. $\ket{n^{(0)}} = \sum_i c_{in}\ket{i^{(0)}}$. 
			
				\begin{solution}[5in]
				\end{solution}
			
			\part Explain why your result means that we can change our basis to any linear combination of the degenerate eigenfunctions without changing the result of our calculation.
			
				\begin{solution}[1.5in]
				\end{solution}
			
		\end{parts}
		\newpage
		
		\question An ``easy'' solution to our degeneracy problem is to require that the numerator, $\braket{m^{(0)}|\hat H^{(1)}|n^{(0)}}$, equals zero whenever $n\neq m$.
		
			\begin{parts}
			
				\part Substitute $\ket{n^{(0)}} = \sum_i c_{in}\ket{i^{(0)}}$ and $\ket{m^{(0)}} = \sum_j c_{jm}\ket{j^{(0)}}$ into $\braket{m^{(0)}|\hat H^{(1)}|n^{(0)}}$ and simplify the expression until it is in terms of $c_{in}$, $c_{jm}$, and $H_{ji}^{(1)}$ (where $H_{ji}^{(1)} = \braket{j^{(0)}|\hat H^{(1)}|i^{(0)}}$ are the matrix elements of the perturbation Hamiltonian).
				
					\begin{solution}[3.5in]
					\end{solution}
				
				\part Remembering that $c_{jm}^* = (\umat{c}^\dag)_{mj}$, how can you rewrite this expression using a matrix product?
				
					\begin{solution}[2in]
					\end{solution}
				
				\part In order for $\braket{m^{(0)}|\hat H^{(1)}|n^{(0)}}$ to be nonzero when $n=m$ but zero otherwise, the resulting matrix must be a \emph{diagonal} matrix.
				
					Explain, in words, what this means about the matrix $\umat{c}$ containing the coefficients $c_{in}$.
				
					\begin{solution}[1in]
					\end{solution}
			
			\end{parts}
	
	\end{questions}