%%%%%%%%%%%%%%%%%%%%%%%%%%%%%%%%%%%%%%%%%
%
% (c) 2018 by Jennifer Laaser
%
% This work is licensed under the Creative Commons Attribution-NonCommercial-ShareAlike 4.0 International License. To view a copy of this license, visit http://creativecommons.org/licenses/by-nc-sa/4.0/ or send a letter to Creative Commons, PO Box 1866, Mountain View, CA 94042, USA.
%
% The current source for these materials is accessible on Github: https://github.com/jlaaser/quantum-exercises
%
%%%%%%%%%%%%%%%%%%%%%%%%%%%%%%%%%%%%%%%%%

\section*{Symmetry with Respect to Exchange\sectionmark{Exercise: Exchange Symmetry}}

	\begin{questions}
	
		\question The permutation operator, $\hat P_{12}$, swaps particles one and 2, e.g. $\hat P_{12} \psi(q_1,q_2) = \psi(q_2,q_1)$.
		
			\begin{parts}
				\part If we apply the permutation operator twice, what should we get? In other words, what is $\hat P_{12}^2 \psi(q_1,q_2)$?
				
					\begin{solution}[1.5in]
					\end{solution}
				
				\part If particles 1 and 2 are indistinguishable, then swapping them should give the same state, up to a constant.  That is, $\psi(q_2,q_1) = c\psi(q_1,q_2)$.
				
				Convince yourself that this means that $\hat P_{12}^2\psi(q_1,q_2) = c^2 \psi(q_1,q_2)$.  In conjunction with your answer to part (a), what does this tell you about the value of $c$?
				
					\begin{solution}[2in]
					\end{solution}
				
				\part Suppose $c=-1$.  What happens if the two particles are at the same spatial coordinates, i.e. $q_1=q_2$?  
				
				\emph{Hint: start with $\psi(q_1,q_1) = -\psi(q_1,q_1)$. If this is true, what can you conclude about the value of $\psi(q_1,q_1)$, and how would you interpret this result?}
				
					\begin{solution}[2.25in]
					\end{solution}
				
			\end{parts}
			
		\contdnewpg
		
		\question When $c=1$, we say that the wavefunction is \emph{symmetric} with respect to exchange, while when $c=-1$, we say it is \emph{antisymmetric} with respect to exchange.%Before we learned about spin, we wrote the ground state of the helium atom as $\psi(1,2) = 1s(1)1s(2)$.
		
			With this in mind, let's consider the ground state of the helium atom, this time including spin:
			
			\begin{parts}
			
				\part One way we could incorporate spin is simply to multiply the spatial wavefunction $1s(1)1s(2)$ by spin functions $\alpha(1)$, $\beta(2)$, etc.  If we do this, four possible descriptions of helium's ground state are:
				
					\begin{enumerate}[itemsep=12pt,topsep=12pt]
						\item $1s(1)\alpha(1)1s(2)\alpha(2)$
						\item $1s(1)\alpha(1)1s(2)\beta(2)$
						\item $1s(1)\beta(1)1s(2)\alpha(2)$
						\item $1s(1)\beta(1)1s(2)\beta(2)$
					\end{enumerate}
					
					%\vspace{6pt}
					In the space to the right, above, identify each of these wavefunctions as symmetric with respect to exchange of electrons 1 and 2, antisymmetric with respect to exchange, or neither.
					
					\vspace{12pt}
				\part Another way we could construct the ground state of the helium atom is using the determinant
					\begin{equation*}
						\psi(1,2) = \frac{1}{\sqrt{2}}\begin{vmatrix}1s(1)\alpha(1) & 1s(1)\beta(1) \\ 1s(2)\alpha(2) & 1s(2)\beta(2)\end{vmatrix}
					\end{equation*}
					Is this wavefunction symmetric or antisymmetric with respect to exchange?
					
					\emph{Hint: you can do this by multiplying the determinant out, but there is also a more elegant way to get to the answer - what happens to the determinant of a matrix if you swap two rows or two columns?}
					
						\begin{solution}[3in]
						\end{solution}
				
			\end{parts}
	
\end{questions}

\stophere