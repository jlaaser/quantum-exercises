%%%%%%%%%%%%%%%%%%%%%%%%%%%%%%%%%%%%%%%%%
%
% (c) 2018 by Jennifer Laaser
%
% This work is licensed under the Creative Commons Attribution-NonCommercial-ShareAlike 4.0 International License. To view a copy of this license, visit http://creativecommons.org/licenses/by-nc-sa/4.0/ or send a letter to Creative Commons, PO Box 1866, Mountain View, CA 94042, USA.
%
% The current source for these materials is accessible on Github: https://github.com/jlaaser/quantum-exercises
%
%%%%%%%%%%%%%%%%%%%%%%%%%%%%%%%%%%%%%%%%%

\section*{The Variational Principle\sectionmark{Exercise: The Variational Principle}}

	In this set of exercises, we'll consider the value of the expectation value of the energy for an arbitrary state.  For these exercises, assume that the $\{\ket{n}\}$ are energy eigenstates, with $\hat H \ket{n} = E_n \ket{n}$.
	
\begin{questions}
	\question Suppose we pick an arbitrary state $\ket{\Psi_t} = \sum_n c_n \ket{n}$.
	
		\begin{parts}
			\part What is the expectation value of the energy, $E_t$, for this state?
			
				\begin{solution}[3in]
				\end{solution}
			
			\part Assuming that $E_0 < E_1 < E_2 < \dots$ (e.g. the energies increase as $n$ increases), what must be true about the relationship between $E_t$ and $E_0$?  Fill in the correct inequality symbol (e.g $=$, $>$, $\geq$, $<$, $\leq$) to express your answer mathematically.
			
				\begin{equation*}
					E_t \,\, \answerbox{30}\,\,  E_0
				\end{equation*}
				
				\vspace{0.1in}
			\part If we knew that $\ket{\Psi_t}$ were \emph{orthogonal} to $\ket{0}$, what would have to be true about $c_0$?  What energy is the lower bound on $E_t$ in this case?
			
				\begin{solution}[2.5in]
				\end{solution}
		\end{parts} 
		\contdnewpg
		
	\question With the relationships from the previous page in mind, we can revisit the helium atom.
	
		\begin{parts}
			\part Suppose we took the same wavefunction that we guessed earlier, but now treat the charge on the nucleus as a \emph{variable} rather than a constant - let's call it $Z'$ rather than $Z$.
			
				Propose a procedure that you could use to find the value of $Z'$ that gives the ``best'' possible solution for the ground state energy of the helium atom.
				
				\begin{solution}[4in]
				\end{solution}
				
			\part Do you think the optimized value of $Z'$ will be greater than or less than the ``true'' value of $Z$?  Why?  
			
			(\emph{You should think about this question in pureley chemical terms - no math or quantum needed here.  Hint: what phenomenon do we talk about when we describe e.g. why some protons have greater chemical shifts in NMR than others?})
			
				\begin{solution}[2.25in]
				\end{solution}
		\end{parts}
	
\end{questions}

\stophere