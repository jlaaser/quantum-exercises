%%%%%%%%%%%%%%%%%%%%%%%%%%%%%%%%%%%%%%%%%
%
% (c) 2018 by Jennifer Laaser
%
% This work is licensed under the Creative Commons Attribution-NonCommercial-ShareAlike 4.0 International License. To view a copy of this license, visit http://creativecommons.org/licenses/by-nc-sa/4.0/ or send a letter to Creative Commons, PO Box 1866, Mountain View, CA 94042, USA.
%
% The current source for these materials is accessible on Github: https://github.com/jlaaser/quantum-exercises
%
%%%%%%%%%%%%%%%%%%%%%%%%%%%%%%%%%%%%%%%%%

\section*{Eigenvalues of the Angular Momentum Operators\sectionmark{Exercise: Angular Momentum Eigenvalues}}

	Now that we know something about what the raising and lowering operators do, let's see if we can use them to learn about the allowed eigenvalues for angular momentum.
	
	\begin{questions}
	
		\question If we apply the raising operator to $\ket{Y_{c,b}}$, we increase the eigenvalue under $\hat M_z$ by $\hbar$, i.e. $\hat M_z \hat M_+ \ket{Y_{c,b}} = (b+\hbar) \hat M_+ \ket{Y_{c,b}}$
		
			\begin{parts}
				\part What happens to the eigenvalue under $\hat M_z$ if we repeatedly apply $\hat M_+$ (i.e. what is $\hat M_z \hat M_+^k \ket{Y_{c,b}}$)?
				
					\begin{solution}[2in]
					\end{solution}
				
				\part We know that $\hat M^2 = \hat M_x^2 + \hat M_y^2 + \hat M_z^2$. Can the eigenvalue of $\hat M_z^2$ ever be larger than the eigenvalue of $\hat M^2$?
				
					\begin{solution}[2in]
					\end{solution}
				
				\part Using your answer to part (b), fill in the boxes, below, with appropriate inequality symbols (e.g. $=$, $>$, $<$, $\geq$, $\leq$, etc.):
				
					\begin{align*}
						c \,\, \answerbox{30} \,\, b^2
					\end{align*}
					or
					\begin{align*}
						-c^{1/2} \,\,\answerbox{30}\,\, b\,\, \answerbox{30}\,\, c^{1/2}
					\end{align*}
			\end{parts}
			
			\contdnewpg
			
		\question Our results from the previous page tell us that there are a maximum and minimum allowed value for $b$. Let's assume the maximum allowed value of $b$ is $b_{max}$, associated with state $\ket{Y_{c,b_{max}}}$, and the minimum allowed value is $b_{min}$, associated with state $\ket{Y_{c,b_{min}}}$.
		
			\begin{parts}
				\part Is $\hat M_+ \ket{Y_{c,b_{max}}}$ an allowed state? Why or why not?
				
					\begin{solution}[2in]
					\end{solution}
				
				\part We can show that $b_{max} = -b_{min}$.  If there are $n$ steps in between, i.e. $b_{max} = b_{min} + n\hbar$, what is $b_{max}$ in terms of $n$?
				
					\begin{solution}[2in]
					\end{solution}
				
				\part If $j = \frac{n}{2}$, what values can $j$ have?
				
					\begin{solution}[1.5in]
					\end{solution}
				
				\part We can also show that $c = b_{max}^2 + \hbar b_{max}$. What must $c$ be, in terms of $j$?
				
					\begin{solution}[1.25in]
					\end{solution}
			\end{parts}
	
	\end{questions}
	
	\stophere