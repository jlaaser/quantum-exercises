%%%%%%%%%%%%%%%%%%%%%%%%%%%%%%%%%%%%%%%%%
%
% (c) 2018 by Jennifer Laaser
%
% This work is licensed under the Creative Commons Attribution-NonCommercial-ShareAlike 4.0 International License. To view a copy of this license, visit http://creativecommons.org/licenses/by-nc-sa/4.0/ or send a letter to Creative Commons, PO Box 1866, Mountain View, CA 94042, USA.
%
% The current source for these materials is accessible on Github: https://github.com/jlaaser/quantum-exercises
%
%%%%%%%%%%%%%%%%%%%%%%%%%%%%%%%%%%%%%%%%%

\section*{Angular Momentum Raising and Lowering Operators\sectionmark{Angular Momentum Ladder Operators}}

	As with the harmonic oscillator, we can define raising and lowering operators for angular momentum, i.e.
	\begin{align*}
		\hat M_+ = \hat M_x + i\hat M_y & & \hat M_- = \hat M_x - i\hat M_y
	\end{align*}
	
	\begin{questions}
		\question Let's start by considering whether or not these raising and lowering operators commute with the other angular momentum operators:
			\begin{parts}
				\part What are $[\hat M^2,\hat M_+]$ and $[\hat M^2,\hat M_-]$?
				
					\begin{solution}[3in]
					\end{solution}
					
				\part What are $[\hat M_z,\hat M_+]$ and $[\hat M_z,\hat M_-]$?
				
					\begin{solution}[2in]
					\end{solution}
				
			\end{parts}
			
		\contdnewpg
		
		\question Now, let's do what we did for the harmonic oscillator, and assume there exists a state $\ket{Y_{c,b}}$ which is an eigenstate of both $\hat M^2$ and $\hat M_z$, such that
			\begin{align*}
				\hat M^2 \ket{Y_{c,b}} &= c \ket{Y_{c,b}} \\
				\hat M_z \ket{Y_{c,b}} &= b \ket{Y_{c,b}}
			\end{align*}
			
			\begin{parts}
				\part Is $\hat M_+ \ket{Y_{c,b}}$ an eigenstate of $\hat M_z$? If so, what is its eigenvalue?
				
					\begin{solution}[3in]
					\end{solution}
				
				\part Is $\hat M_+ \ket{Y_{c,b}}$ an eigenstate of $\hat M^2$? If so, what is its eigenvalue?
				
					\begin{solution}[2.5in]
					\end{solution}
				
				\part Given these results, why might we call $\hat M_+$ a ``raising operator'' for angular momentum?
			\end{parts}
	\end{questions}

