%%%%%%%%%%%%%%%%%%%%%%%%%%%%%%%%%%%%%%%%%
%
% (c) 2018 by Jennifer Laaser
%
% This work is licensed under the Creative Commons Attribution-NonCommercial-ShareAlike 4.0 International License. To view a copy of this license, visit http://creativecommons.org/licenses/by-nc-sa/4.0/ or send a letter to Creative Commons, PO Box 1866, Mountain View, CA 94042, USA.
%
% The current source for these materials is accessible on Github: https://github.com/jlaaser/quantum-exercises
%
%%%%%%%%%%%%%%%%%%%%%%%%%%%%%%%%%%%%%%%%%

\section*{The Schr\"odinger Equation for the Hydrogen Atom\sectionmark{Exercise: The Hydrogen Atom}}

	In this set of exercises, we will look at the Schr\"odinger equation for the Hydrogen-like atom.

	\begin{questions}
	
		\question A ``hydrogen-like'' atom consists of a nucleus of charge $+Ze$ and mass $m_n$ and an electron of charge $-e$ and mass $m_e$.
			\begin{parts}
				\part If the nucleus is at coordinates $\vec{R_n} = (x_n,y_n,z_n)$ and the electron is at $\vec{R_e} = (x_e,y_e,z_e)$, what is the total kinetic energy for this system?
				
					\begin{solution}[2in]
					\end{solution}
				
				\part  If the potential energy is given by $V(x_n,y_n,z_n,x_e,y_e,z_e)$, what is the Schr\"odinger equation for this system?
				
					\begin{solution}[2in]
					\end{solution}
				
				\part We can simplify the potential energy term because the potential is purely electrostatic, and depends only on the \emph{distance} $r$ between the nucleus and the electron, not their absolute positions.  
				
				If the electrostatic potential energy for charges $Q_1$ and $Q_2$ separated by distance $r$ is $\frac{Q_1 Q_2}{4\pi\epsilon_0 r}$, what is $V(r)$ for the hydrogen-like atom?
				
					\begin{solution}[2in]
					\end{solution}
			\end{parts}
			
		\newpage
		\question If we switch to center-of-mass coordinates, we can separate out the translational motion of the atom from the relative motion of the electron and nucleus.
		
			\begin{parts}
				\part The \emph{relative} Hamiltonian for the hydrogen-like atom is given by
					\begin{equation*}
						\hat H_{rel} = \frac{-\hbar^2}{2\mu}\left(\genpar{}{r}{2} + \frac{2}{r}\genpar{}{r}{}\right) + \frac{\hat L^2}{2\mu r^2} - \frac{Ze^2}{4\pi\epsilon_0 r}
					\end{equation*}
					Try applying this Hamiltonian to $\Psi(r,\theta,\phi) = R(r)Y_l^m(\theta,\phi)$. How can you simplify this equation?  \emph{Hint: remember that $\hat L^2 Y_l^m(\theta,\phi) = l(l+1)\hbar^2Y_l^m(\theta,\phi)$}.
				
					\begin{solution}[3in]
					\end{solution}
					
				\part How many quantum numbers should we need to describe the entire spatial wavefunction? \emph{Hint: We know from our discussion of the spherical harmonics that we will need two quantum numbers to describe the angular part of the wavefunction ($l$ and $m$). How many more do you think you will get from solving the radial part?}
				
					\begin{solution}[1.5in]
					\end{solution}
				
				\part Will the radial wavefunctions depend on $l$? $m$?  How about the energies - which quantum numbers do you think they will depend on?
				
					\begin{solution}[1.5in]
					\end{solution}
			\end{parts}
	\end{questions}

