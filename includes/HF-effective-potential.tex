%%%%%%%%%%%%%%%%%%%%%%%%%%%%%%%%%%%%%%%%%
%
% (c) 2018 by Jennifer Laaser
%
% This work is licensed under the Creative Commons Attribution-NonCommercial-ShareAlike 4.0 International License. To view a copy of this license, visit http://creativecommons.org/licenses/by-nc-sa/4.0/ or send a letter to Creative Commons, PO Box 1866, Mountain View, CA 94042, USA.
%
% The current source for these materials is accessible on Github: https://github.com/jlaaser/quantum-exercises
%
%%%%%%%%%%%%%%%%%%%%%%%%%%%%%%%%%%%%%%%%%

\section*{The Effective Potential in the Hartree Method\sectionmark{Exercise: Hartree Method Effective Potential}}
	
	In this exercise, we'll think about the physical meaning of the effective potential 
			\begin{equation*}
				V_{eff}(\vec{r_2}) = \int d\tau_1 \, \phi^*(1)\frac{e^2}{4\pi \epsilon_0 r_{12}} \phi(1)
			\end{equation*}

	\begin{questions}
	
		\question Consider an electron with wavefunction $\phi(1)\equiv \phi(\vec{r_1})$:
			
			\begin{parts}
				\part What is the \emph{probability density} for this electron?
				
					\begin{solution}[2in]
					\end{solution}
				
				\part What is the probability of finding the electron in a specific volume element $d\tau_1$ near position $\vec{r_1}$?
				
					\begin{solution}[2in]
					\end{solution}
				
				\part On average, how much charge is in this volume element $d\tau_1$?  (Remember that a single electron has charge $e$.)
				
					\begin{solution}[2in]
					\end{solution}
				
				\contdnewpg
				
				\part Suppose there is another electron (let's call it electron 2) at position $\vec{r_2}$.  What is the potential energy for the interaction between electron 2 and the piece of electron 1's charge in volume element $d\tau_1$ near position $\vec{r_1}$?
				
					\begin{solution}[2in]
					\end{solution}
				\part What is the total potential energy for electron 2 at position $\vec{r_2}$ interacting with electron 1 (i.e. what do we get if we integrate over all possible positions of electron 1)?
				
					\begin{solution}[2in]
					\end{solution}
				
				\part Compare your answer to part (e) with the expression for $V_{eff}$ at the beginning of this exercise. Explain, in your own words, why we might say that $V_{eff}$ is the potential that electron 2 ``sees'' when it interacts only with the average charge distribution of electron 1.
				
					\begin{solution}[3in]
					\end{solution}
			\end{parts}
	
\end{questions}

\stophere