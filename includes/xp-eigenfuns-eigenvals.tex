%%%%%%%%%%%%%%%%%%%%%%%%%%%%%%%%%%%%%%%%%
%
% (c) 2018 by Jennifer Laaser
%
% This work is licensed under the Creative Commons Attribution-NonCommercial-ShareAlike 4.0 International License. To view a copy of this license, visit http://creativecommons.org/licenses/by-nc-sa/4.0/ or send a letter to Creative Commons, PO Box 1866, Mountain View, CA 94042, USA.
%
% The current source for these materials is accessible on Github: https://github.com/jlaaser/quantum-exercises
%
%%%%%%%%%%%%%%%%%%%%%%%%%%%%%%%%%%%%%%%%%

\section*{Position and Momentum: Eigenfunctions and Operators\sectionmark{Exercise: Position and Momentum}}

	Two particularly useful operators in quantum mechanics are the position operator, $\hat x$, and the momentum operator, $\hat p$.
	
	In this set of exercises, we will consider what these operators and their eigenfunctions look like in the position representation.
	
	\vspace{0.25in}
	
	\begin{questions}
		\question An eigenstate of the position operator is one where the probability of finding the particle is nonzero at only one position.
		
			\begin{parts}
				\part Predict what the probability density for a particle found at $x=x_0$ should look like:
				
\vspace{0.5in}					\centerline{\includegraphics[width=0.5\textwidth]{includes/xp-eigenfuns-eigenvals-FIGURES/x0_axes}}		
\vspace{0.3in}		
				
				\part The eigenvalue of this position eigenstate should be the particle's position, e.g. $x_0$. Which of the following operators would be most useful for extracting this eigenvalue?
				
				\begin{align*}
					\hat \Omega \psi(x) &= \psi(x) + x  && (\hat\Omega = \text{``add x'')}\\
					\hat \Omega \psi(x) &= x\psi(x)&& (\hat\Omega = \text{``multiply by x'')}\\
					\hat \Omega \psi(x) &= |\psi(x)|^2&& (\hat\Omega = \text{``square the function'')}\\
					\hat \Omega \psi(x) &= \frac{d}{dx}\psi(x)  && (\hat\Omega = \text{``take the derivative'')}
				\end{align*}
				
				Briefly explain your choice:
				
				\begin{solution}[1.5in]
				\end{solution}
			
			\end{parts}
			
			
		\contdnewpg
		\newpage 
		\item De Broglie told us that a particle with momentum $p$ has wavelength $\lambda = \frac{h}{p}$.
		
			\begin{parts}
				\part One useful function that has a wave-like form is $f(x) = e^{ikx} = \cos(kx) + i\sin(kx)$.  What is the wavelength of this function in terms of $k$ (i.e. for what value of $\lambda$ does $f(x+\lambda) = f(x)$)?
				
					\begin{solution}[2in]
					\end{solution}
				
				\part Set this wavelength equal to de Broglie's value for $\lambda$. How is $k$ related to $p$?
				
					\begin{solution}[2in]
					\end{solution}
				
				
				\part What operator might you be able to apply to the function $e^{ikx}$ to extract the momentum eigenvalue $p$?
				
					\begin{solution}[3in]
					\end{solution}
				
			\end{parts}
	\end{questions}
	
	\stophere