%%%%%%%%%%%%%%%%%%%%%%%%%%%%%%%%%%%%%%%%%
%
% (c) 2018 by Jennifer Laaser
%
% This work is licensed under the Creative Commons Attribution-NonCommercial-ShareAlike 4.0 International License. To view a copy of this license, visit http://creativecommons.org/licenses/by-nc-sa/4.0/ or send a letter to Creative Commons, PO Box 1866, Mountain View, CA 94042, USA.
%
% The current source for these materials is accessible on Github: https://github.com/jlaaser/quantum-exercises
%
%%%%%%%%%%%%%%%%%%%%%%%%%%%%%%%%%%%%%%%%%

\section*{Generalizing the Free Particle\sectionmark{Exercise: Generalizing the Free Particle}}

	In this set of exercises, we will explore how to think about an arbitrary wavefunction on a constant potential in terms of the free particle momentum eigenstates.

	\begin{questions}
	
		\question Let's start by considering how the wavefunction in the position representation is related to the wavefunction in the momentum representation.
		
			\begin{parts}
				\part The wavefunction in the position representation is given by $\Psi(x) = \braket{x\underset{\uparrow}|\Psi}$.
		
					Insert a complete set of momentum eigenstates at the arrow and simplify to find an expression for $\Psi(x)$ in terms of $\Psi(k)$.
		
				\begin{solution}[2.5in]
				\end{solution}
				
				\part Use the same process to find an expression for $\Psi(k) = \braket{k|\Psi}$ in terms of $\Psi(x)$.
				
					\begin{solution}[2.5in]
					\end{solution}
					
				\part Do you notice anything interesting about these relationships? If you know the formal name for this type of transformation, go ahead and write it down:
			
			\end{parts}
		\vspace{1in}
		\contdnewpg
		\question Now, let's consider the time-dependence of a particle on a constant potential.
		
			\begin{parts}
				\part If a system starts in a pure momentum eigenstate, i.e. $\Psi(x,t=0) = \phi_k(x)$, what is its \emph{time-dependent} wavefunction $\Psi(x,t)$?
				
					\emph{You'll probably find it useful to remember that on a constant potential $V(x)=0$, the momentum eigenstate with quantum number $k$ has wavefunction $\phi_k(x) = e^{ikx}$ and energy $E_k = \frac{\hbar^2 k^2}{2m}$.}
					
					\begin{solution}[3in]
					\end{solution}
			
				\part For a discrete set of eigenstates, a particle with $\Psi(x,t=0) = \sum_n c_n \phi_n(x)$ has time-dependent wavefunction $\Psi(x,t) = \sum_n c_n e^{-iE_nt/\hbar} \phi_n(x)$.
				
					By analogy, what do you think the time-dependent wavefunction should be for a particle with $\Psi(x,t=0) = \int_{-\infty}^{\infty} dk\, c(k) \phi_k(x)$?
				
			\end{parts}
	\end{questions}
	
	\begin{solution}[3.25in]
	\end{solution}
	
\stophere