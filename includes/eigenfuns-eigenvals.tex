%%%%%%%%%%%%%%%%%%%%%%%%%%%%%%%%%%%%%%%%%
%
% (c) 2018 by Jennifer Laaser
%
% This work is licensed under the Creative Commons Attribution-NonCommercial-ShareAlike 4.0 International License. To view a copy of this license, visit http://creativecommons.org/licenses/by-nc-sa/4.0/ or send a letter to Creative Commons, PO Box 1866, Mountain View, CA 94042, USA.
%
% The current source for these materials is accessible on Github: https://github.com/jlaaser/quantum-exercises
%
%%%%%%%%%%%%%%%%%%%%%%%%%%%%%%%%%%%%%%%%%

\section*{Eigenfunctions and Eigenvalues\sectionmark{Eigenfunctions and Eigenvalues}}

	\begin{questions}
		\question Assume $\ket{f}$ is an eigenket of $\hat\Omega$ with eigenvalue $\omega$, i.e.
			\begin{equation*}
				\hat\Omega \ket{f} = \omega\ket{f}
			\end{equation*}
			
			\begin{parts}
			
				\part What is $\braket{f|\hat\Omega|f}$?
					\begin{solution}[1in]
					\end{solution}
					
				\part What is $\braket{f|\hat\Omega|f}^*$?
					\begin{solution}[1in]
					\end{solution}
					
				\part If $\hat \Omega$ is Hermitian, then your answers to the preceding two parts have to be equal.  
				
					What does that tell you about the values of $\omega$ and $\omega^*$?
				
					\begin{solution}[1.5in]
					\end{solution}
					
				\part If a number is equal to its complex conjugate, then can the number have a non-zero imaginary part? Why or why not?
				
					\begin{solution}[1.5in]
					\end{solution}
					
				\part What does that tell you about the value of $\omega$?
				
					\begin{solution}[1in]
					\end{solution}
			\begin{flushright}(Continued on back of page $\rightarrow$)	\end{flushright}
					
			\end{parts}
			
		\newpage
		\question The Kronecker delta, $\delta_{ij}$, equals 1 when $i=j$ and is 0 when $i\neq j$.
		
			\begin{parts}
				\part Write out the first few terms of the sum $\sum_j \delta_{3j} c_j$.  What does it simplify to?
				
					\begin{solution}[1in]
					\end{solution}
			
				\part More generally, what do you think $\sum_j \delta_{ij} c_j$ will simplify to?
				
					\begin{solution}[1in]
					\end{solution}
				
			\end{parts}
			
		\question We can always choose the eigenstates $\{\ket{i}\}$ of a Hermitian operator such that the overlap of an eigenstate with itself equals 1 and the overlap of an eigenstate with a different eigenstate equals 0, i.e. such that
			\begin{equation*}
				\braket{i|j} = \delta_{ij}
			\end{equation*}
			
			\begin{parts}
				\part If we write $\ket{\Psi}$ as a linear combination of these eigenstates, i.e.
					\begin{equation*}
						\ket{\Psi} = \sum_j c_j\ket{j}
					\end{equation*}
					what is $\braket{i|\Psi}$?
				
					\begin{solution}[2in]
					\end{solution}
					
				\part Substitute your answer back into $\ket{\Psi} = \sum_j c_j\ket{j}$.
				
					Why might we say that $\sum_i \ket{i}\bra{i}$ is just equivalent to multiplying by 1?
			\end{parts}
			
	\end{questions}

