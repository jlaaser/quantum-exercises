\section*{Describing States and Operators with Matrices\sectionmark{Intro to Matrix Mechanics}}

	In this set of exercises, we will look at how vectors and matrices can be used to describe and calculate useful properties of quantum states and operators.

	\begin{questions}
	
		\question
		\begin{parts}
			\part Calculate the following matrix product:
			\begin{equation}
				\begin{bmatrix} c_1^* & c_2^* \end{bmatrix}
				\begin{bmatrix} A_{11} & A_{12} \\ A_{21} & A_{22} \end{bmatrix}
				\begin{bmatrix} c_1 \\ c_2 \end{bmatrix} \nonumber
			\end{equation}
			
				\begin{solution}[2.5in]
				\end{solution}
			
			\part Calculate $\exptval{A}=\braket{\psi|\hat A|\psi}$ where $\ket{\psi} = c_1 \ket{1} + c_2 \ket{2}$
			
				\begin{solution}[2.5in]
				\end{solution}
			
			\part Given your answers to the previous two questions, why might we call $\braket{n|\hat A|m}$ a ``matrix element'' of $\hat A$?
			
				\begin{solution}[1.3in]
				\end{solution}
			
			\contdnewpg
		\end{parts}
			
		\question 
		\begin{parts}
			\part If we calculate matrix elements of $\hat A$ using one basis set (e.g. the particle-in-a-box eigenstates), will it be the same as if we use a different basis set (e.g. the harmonic oscillator eigenstates)?  Why or why not?
			
				\begin{solution}[2.5in]
				\end{solution}
				
			\part Explain why the following statement is correct: ``If $\ket\Psi = \sum_n c_n \ket{n}$, and we clearly state what basis states the $\{\ket{n}\}$ represent, then the only thing we need to fully describe $\ket\Psi$ are the coefficients $c_n$.''
			
				\begin{solution}[2.5in]
				\end{solution}
				
			\part Write an analogous statement about how we might be able to describe $\hat A$ solely in terms of its matrix elements.
			
				\begin{solution}[2in]
				\end{solution}
		\end{parts}
		
		%	\question If the matrix element $A_{nm} = \braket{n|\hat A|m}$, and $\hat A$ is Hermitian, then what is $A_{mn}$?
		
		%		\begin{solution}[2.5in]
		%		\end{solution}
	\end{questions}
	
