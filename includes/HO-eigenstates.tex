%%%%%%%%%%%%%%%%%%%%%%%%%%%%%%%%%%%%%%%%%
%
% (c) 2018 by Jennifer Laaser
%
% This work is licensed under the Creative Commons Attribution-NonCommercial-ShareAlike 4.0 International License. To view a copy of this license, visit http://creativecommons.org/licenses/by-nc-sa/4.0/ or send a letter to Creative Commons, PO Box 1866, Mountain View, CA 94042, USA.
%
% The current source for these materials is accessible on Github: https://github.com/jlaaser/quantum-exercises
%
%%%%%%%%%%%%%%%%%%%%%%%%%%%%%%%%%%%%%%%%%

\section*{Eigenvalues and Eigenstates of the Harmonic Oscillator\sectionmark{Harmonic Oscillator Eigenstates}}

	\begin{questions}
		\question Let's start by considering the range of values that $n$ can take.
			\begin{parts}
				\part We know that for a quantum state $\alpha$, $\braket{\alpha|\alpha}\geq 0$.  What does this tell you about the minimum possible value of $\braket{n|\adag \hat a|n}$?
			
				\emph{(Hint: start by defining $\ket\alpha = \hat a\ket n$. What is $\bra{\alpha}$, and what do you get if you put these pieces together?)}
			
				\begin{solution}[3in]
				\end{solution}
			
				\part We also showed on the previous page that $\braket{n|\adag \hat a|n} = n$.  What does this tell you about the minimum possible value of $n$?
				
				\begin{solution}[1.5in]
				\end{solution}
				
				\part What happens if you try to calculate $\hat a \ket{0.5}$?  What do you think that means about whether $\ket{0.5}$ is an allowable quantum state for the harmonic oscillator?
				
					\begin{solution}[1.75in]
					\end{solution}
				\contdnewpg
				
				\part What happens if you calculate $\hat a \ket{1.5}$?  Given your answer to part (c), what do you think this means about whether $\ket{1.5}$ is an allowable quantum state?
				
					\begin{solution}[2in]
					\end{solution}
				
				\part Explain why this reasoning suggests that integers are the only allowable quantum numbers for the harmonic oscillator.
				
					\begin{solution}[3in]
					\end{solution}
			
			\end{parts}
			
		\question \begin{parts}
		
			\part Suppose we knew what the eigenstate $\ket{0}$ was.  What might you do to find eigenstate $\ket{1}$?  How about $\ket{2}$, or $\ket{3}$?  How about $\ket{n}$?
		
			%\part Suppose we knew a functional form for $\braket{\xi|0} = \phi_0(\xi)$.  What might you do to find $\phi_1(\xi)$?
			
			\end{parts}
	\end{questions}
	