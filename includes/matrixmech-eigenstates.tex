%%%%%%%%%%%%%%%%%%%%%%%%%%%%%%%%%%%%%%%%%
%
% (c) 2018 by Jennifer Laaser
%
% This work is licensed under the Creative Commons Attribution-NonCommercial-ShareAlike 4.0 International License. To view a copy of this license, visit http://creativecommons.org/licenses/by-nc-sa/4.0/ or send a letter to Creative Commons, PO Box 1866, Mountain View, CA 94042, USA.
%
% The current source for these materials is accessible on Github: https://github.com/jlaaser/quantum-exercises
%
%%%%%%%%%%%%%%%%%%%%%%%%%%%%%%%%%%%%%%%%%

\section*{Eigenvalues, Eigenstates, and Eigenvectors\sectionmark{Exercise: Eigenstates in Matrix Mechanics}}

	In this set of exercises, we will think about how to express eigenvalue-eigenstate relationships using the matrix formalism.
	
	\begin{questions}
		\question 
		\begin{parts}
			\part Using matrix notation, what needs to be true if $\ket\Psi = \sum_n c_n\ket n$ is an eigenstate of $\hat\Omega$ with eigenvalue $\omega$?
		
				\emph{Hint: in Dirac notation, $\ket\Psi$ is an eigenstate of $\hat\Omega$ if $\hat\Omega \ket\Psi = \omega\ket\Psi$. What analogous expression do you get if you simply replace $\ket\Psi$ and $\hat\Omega$ by their matrix-notation equivalents?}
			
				\begin{solution}[2in]
				\end{solution}
			
			\part Using your answer to the preceding question, fill in the blank to complete the following statement:
			
				``If $\ket\Psi$ is an eigenstate of $\hat\Omega$, then the vector $\uvec{c} = \begin{pmatrix}c_1 \\ c_2 \\ c_3 \\ \vdots\end{pmatrix}$ must be an $\answerbox{120}$ of the matrix $\umat\Omega$.''
				
					\vspace{0.25in}
			
			\part Suppose we had a matrix $\umat\Omega$ containing the matrix elements of $\hat\Omega$ in some basis $\{\ket n\}$.
			
				What steps could you take to find the eigenstates of $\hat\Omega$ in terms of the $\{\ket n\}$ basis?
				
				\begin{solution}[2.8in]
				\end{solution}
				
				\contdnewpg
			
		\end{parts}
		
		\question Often, it is useful to be able to ``transform'' the matrix corresponding to an operator (most often the Hamiltonian) from one basis set to another.
		
			\begin{parts}
				\part Suppose the $\{\ket n\}$ are the true eigenstates of $\hat H$, such that $\hat H\ket n = E_n \ket n$.  Using Dirac notation, what are the matrix elements of $\hat H$ in this basis, i.e. what is $H_{nm}$?
				
					\begin{solution}[1.25in]
					\end{solution}	
				
				\part Now, assume we only know the matrix elements of $\hat H$ in some other basis, $\{\ket i\}$.  What are the matrix elements in this basis (i.e. what is $H_{ij}$?
				
					\begin{solution}[1.25in]
					\end{solution}	
				
				\part If we take our expression from part (a) and insert a complete set of $\{\ket{i}\}$ eigenstates at each ``$\vert$'', we get
				\begin{equation*}
					H_{nm} = \sum_i \sum_j \braket{n|i} \braket{i|\hat H|j}\braket{j|m}
				\end{equation*}
				How would you interpret the $\braket{n|i}$ and $\braket{j|m}$ terms?
				
					\begin{solution}[1.5in]
					\end{solution}
				
				\part If we compiled all of the $\braket{n|i}$ values in a matrix $\umat{S}$ with matrix elements $S_{ni} = \braket{n|i}$, what matrix expression could you use to calculate $\umat{H}_{\{\ket{n}\}}$ (the Hamiltonian in terms of the $\{\ket{n}\}$ eigenstates) from $\umat{H}_{\{\ket{i}\}}$ (the Hamiltonian in terms of the $\{\ket{i}\}$ states)?
				
					\begin{solution}[1.5in]
					\end{solution}
			\end{parts}
	\end{questions}

\stophere